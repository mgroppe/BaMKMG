\documentclass[extern,palatino]{cgBA}
\usepackage{setspace}
%\onehalfspacing
\doublespacing

\author{Martin Groppe und Malte Kremer}
\title{Entwicklung eines AR-Rollenspiels für Android}
\zweitgutachter{...}
\externLogo{7.46cm}{logos/UniLogoNeu}
\externName{DIN: NewTechnologies}

\begin{document}

% Umschalten der Sprache (für englische Rubrikbezeichnungen etc.)
%\selectlanguage{english}


\maketitle

\newpage

\pagenumbering{roman}
\tableofcontents
\clearpage         % oder \cleardoublepage bei zweiseitigem Druck
% \listoffigures   % fuer ein eventuelles Abbildungsverzeichnis
% \clearpage
\pagenumbering{arabic}


\section{abstract}

Pokémon Go wurde in den ersten drei Monaten nach dem Release 500 Millionen mal heruntergeladen, bleibt aber als Hybrid aus Rollen- und augmented reality-Spiel eher die Ausnahme. 
\\Ziel dieser Arbeit ist es daher, vergleichbare Produkte zu analysieren und aus der Analyse einen Prototypen für ein AR-Rollenspiel zu entwickeln. Anschließend wird dieser von Testpersonen darauf geprüft, ob er Spaß macht und das Potenzial hätte, langfristig zu fesseln um allgemeine Erkenntisse über die Qualität des Prototypen zu gewinnen.
\section{abstract (english)}
Pokémon Go had enormous success, was downloaded 500 million times in the first three months. It is however a rare case of a combination of Augmented reality and role-playing game features.
\\This thesis intends to develop a role-playing game that uses Google Maps features and is fun and engaging. The implementation will then be tested and evaluated.
\newpage
\section{Motivation}
Spätestens seit dem massiven Erfolg von Pokémon Go sind Smartphones als Spieleplatform nicht mehr wegzudenken. Dennoch bleibt Pokémon Go eher eine Ausnahme insofern, dass die Verknüpfung der AR-Möglichkeiten des Smartphones selten mit klassischen Spielkonzepten kombiniert wird. Dabei hat Pokémon Go gezeigt, dass es relativ einfach ist, die reale Welt in Rollenspiele einzubauen. \\Daher haben wir es uns zum Ziel gesetzt zu erforschen, inwieweit es möglich ist, in einem relativ kurzen Zeitraum und mit geringem Budget einen Prototyp zu entwickeln, der AR-Elemente beinhalten, kurzfristig Spaß machen soll und das Potenzial haben soll langfristig zu fesseln.
\newpage
\section{Grundlagen}
Da das Spiel auf Android laufen soll, wird Interface und Basis-Programm mit java programmiert. Als Entwicklungsumgebung für den java-Code wird Google's hauseigenes Android Studio verwendet, da es gut dokumentiert und weit verbreitet ist.
\\Die Kämpfe hingegen werden mit der Unity-game-engine erstellt, sodass für die Erstellung der Grafik nur ein Bildbearbeitungsprogramm benötigt wird. Dazu wurde Adobe Photoshop verwendet.
\subsection{android}
\subsubsection{AR-Funktionen von Android}
\newpage
\subsection{Unity}
Unity ist eine 2005 von von Unity Technologies SF veröffentlichte Game engine und ist seitdem zunehmend erfolgreich. %laut wikipedia nutzen 47% der Spiele auf mobilen Geräten unity.
Sie wird von Indie-Entwicklern wie William Chyr Studios wie von Publisher-abhängigen Entwicklungsstudios wie Square Enix Montreal benutzt und ermöglicht Entwicklung ebenso für Smartphone-Betriebssysteme wie iOS, Android, Windows Phone wie auch für Mac, Windows, die Playstation 4 oder den 3DS. 
\\Insgesamt sind mit Unity über 238.000 Spiele für den mobilen Markt (Quelle: unity analytics) und etwa 34\% der Top 1000 mobilen Spiele entwickelt worden, es ist die meistbenutzte nicht-inhouse-Engine. Allein im dritten Quartal 2016 hatte Unity über fünf Milliarden downloads. Außerdem ist es benutzerfreundlich, gut dokumentiert und die große Verbreitung machen den Einstieg einfach.
\\Unity bietet die Möglichkeit in 2D sowie in 3D zu programmieren, hat Automatismen für die Erstellung von Animationen aus Spritesheets und bietet übersichtliche objektorientierte Verknüpfungen von Scripts und Game-Objekten. Unitys' Scripts müssen entweder in C\# oder in JavaScript geschrieben werden.
\newpage
\subsection{Photoshop}
Photoshop ist ein Bildbearbeitungsprogram, das 1988 von Adobe entwickelt wurde und seitdem mehrfach weiterentwickelt wurde. Mit Photoshop werden Bilder auf Pixel-basis bearbeitet. Dies ermöglicht relativ einfache Erstellung von sogenannten Sprites, Bildern die als Basismodelle für 2D-Figuren dienen.
\\
Photoshop ist in unter Game-Artists weit verbreitet und wird für die Erstellung von Storyboards, Sketches und Sprites benutzt. Animation Career Review listet es als die essenziellste Software für Künstler und Designer, Developer hat es in ihrer nicht-sortierten Top acht Liste, bloopanimation empfiehlt es ebenfalls für 2d-Animation und bezeichnet es als "großartige Wahl (great choice). 2005 war Photoshops' Marktanteil zwischen 60\% und 70\% im Bereich Bildbearbeitungssoftware für Rastergrafik (Bundeskartellamt, %http://www.bundeskartellamt.de/SharedDocs/Entscheidung/DE/Entscheidungen/Fusionskontrolle/2005/B7-162-05.pdf?\_\_blob=publicationFile\\&v=3
), 2010 hatte Photoshop über 10 Millionen Nutzer weltweit.
\\
Außerdem ist es gut dokumentiert und leicht zu lernen (eine Google-Suche nach Photoshop Tutorial lieferte rund 14.500.000 Ergebnisse). All das machte es zu einer guten Wahl für die Erstellung der Grafik.
\newpage

\section{Recherche und Konzeption}
\subsection{Rollenspiel und AR-Hybriden auf Android}
\subsubsection{Pokémon Go}
Pokémon Go ist ein freemium/free-to-play Smartphone-Spiel für iOS und Android. Es wurde von Niantic entwickelt und im Juli 2016 veröffentlicht und benutzt Google Maps sowie situational das Gyroskop und die Kamera für AR-Funtionen. 
\\Pokémon Go ist bis heute das erfolgreichste Spiel das für Smartphones erschienen ist, mit über 650 Millionen Downloads (Stand 27.02.2017) und über 500 Millionen Downloads in den ersten zwei Monaten.%http://archive.is/XCgwL
 2016 haben gleichzeitig an einem Tag 23 Millionen Nutzer gespielt.
\\
Im Folgenden wird grob die Funktionsweise von Pokémon Go erklärt. Pokémon Go benutzt GPS, um die Position der Spieler zu erfassen und auf einer Google Maps ähnlichen Karte anzuzeigen. Niantics' Server spawned Pokémon (Monster, die man fangen, trainieren und zum kämpfen benutzen kann) an mehr oder weniger zufälligen realen GPS-Positionen. Dabei berücksichtigt der Server eine Reihe von Faktoren, Pokémon vom Typ Wasser werden z.B. eher in der Nähe von Flüssen, Seen und Brunnen erschaffen, während Pokémon vom Typ Geist eine erhöhte Chance haben, bei Nacht aufzutauchen. Wenn der Spieler nah genug an ein Pokémon heran kommt, kann der Spieler das Pokémon sehen und auf es tippen, um einen Bildschirm aufzurufen, in dem sie versuchen können, es mit einem Pokéball zu fangen. Sollte es dem Spieler nicht gelingen, dass Pokémon innerhalb einer bestimmten Zeit zu fangen, entkommt es.
\\Außerdem gibt es an festgelegten Orten sogenannte Poké-Stops, an denen Spieler Items wie die vorher erwähnten Pokébälle bekommen können. Zusätzlich gibt es Arenen, an denen Spieler gegen die Pokémon von anderen Spielern kämpfen können um Items zu bekommen. Dies führt dazu, dass Spieler häufig in der realen Welt bestimmte Routen ablaufen und sich an festgelegten Orten immer wieder begegnen, was mehr oder weniger automatisch zu neuen Kontakten und Freundschaften führt. Neben den realen Kontakten mit denen man das Spiel spielen kann und den Orten, die man besucht, sorgt auch die Kamera-Anzeige in der Pokémon vor einem realen Hintergrund angezeigt werden und die Anpassung von Pokémon in Gebieten für eine Verknüpfung von realer und virtueller Welt. 
\\Das Spiel herunterladen ist kostenlos, im Shop kann man z.B. Pokébälle oder Inventar-Erweiterungen für eine Ingame-Währung kaufen, die man auch mit realem Geld kaufen kann. Nutzer die reales Geld ausgeben haben dementsprechend Vorteile, die einen schnelleren Fortschritt ermöglichen. Laut Schätzungen hat Pokémon Go 2016 etwa 950 Millionen US-Dollar Einnahmen erzeugt.%https://venturebeat.com/2017/01/17/pokemon-go-generated-revenues-of-950-million-in-2016/
\\Auch wenn Pokémon Go alle Smartphone-Spiel-Rekorde gebrochen hat, gibt es doch einige Kritik. Die Interaktionsmöglichkeiten mit anderen Spielern hielten sich stark in Grenzen, das Monetisierungs-Modell war recht unterentwickelt und es gab kein wirklich forderndes Element. Arena-Kämpfe erwiesen sich als recht repetetiv und anspruchslos und echtes Player vs Player-Gameplay, bei dem beide Seiten aktiv Einfluss auf das Kampfgeschehen nehmen können gibt es nicht. Es gibt auch keinen anderen fordernden Endgame-Inhalt, sodass viele Pokémon erst trainierten um dann festzustellen, dass sie nicht viel mit den Pokémon machen konnten. %http://archive.is/bFBtX
\newpage
\subsubsection{Parallel Kingdom}
Parallel Kingdom war ein Spiel, das Elemente aus MMORPG-, Geocache-Spielen und von Real-time browser-based MMOs wie OGame beinhaltete. Es wurde 2008 von PerBlue released und November 2016 abgeschaltet. 
\\Das Spiel mischte verschiedene typische MMO-Elemente wie Ressourcen und Items sammeln und tauschen, Dungeons erforschen, Monster bekämpfen und Fähigkeiten steigern mit einer realen Weltkarte, auf der man sich (anders als in Pokémon Go oder in unserem Spiel) auch virtuell fortbewegen konnte, der Möglichkeit Territorium zu kontrollieren und OGame Features wie Gebäuden die stetigen Ertrag bringen und PvP-Kriegen um Gebiete.\\Das Spiel war mit über 1.000.000 Spielern recht erfolgreich, wurde mehrfach für Awards nominiert und hatte mit acht Jahren eine für ein Android Spiel sehr lange Lebensdauer.  %https://www.nowgamer.com/parallel-kingdom-interview-when-mmo-meets-gps/
Die Abschaltung war laut Entwickler hauptsächlich wegen der veralteten Technik nötig. %http://www.parallelkingdom.com/community/update-hut.aspx#302
Die lange Zeit, über die das Spiel lief zeigt, dass die Konzepte gut ineinander griffen und es einiges an erzielbarem Fortschritt gab. Leider waren viele der Konzepte im Kern recht simpel, Kämpfe bestanden häufig aus nebeneinander laufen und anschließend den Gegner wiederholt anklicken (ohne ernsthafte Strategie) und wurden oft durch Items und Level entschieden. Leider kamen wir nicht dazu, die großen PvP-Funktionen zu testen, da das Spiel zum Start dieser Arbeit lange über seine Hochphase hinaus und kurz vor der Abschaltung war.
\subsection{Rollenspielen auf anderen Plattformen und Portierungen}
Rollenspiele auf dem PC-Markt setzen häufig andere Prioritäten als Smartphone Titel wie die beiden obengenannten. Viele klassischen Rollenspiele legen großen Wert auf Dialoge und ihre Geschichte. Für Skyrim wurden 83 Schauspielern für Tonaufnahmen eingestellt, fast so viele wie Entwickler, Horizon's Szenario ist eines von ursprünglich 40 verschiedenen und für die finale Story wurden etwa 20 verschiedene Entwürfe gemacht. Dragon Age: Inquisition hat über 140 Sprecher und zehn Storywriter.  %http://www.imdb.com/title/tt3763912/fullcredits?ref_=tt_cl_sm#cast %http://www.statisticbrain.com/skyrim-the-elder-scrolls-v-statistics/
%https://en.wikipedia.org/wiki/Horizon_Zero_Dawn#Development
Auch große Konsolenprojekte wie moderne Ableger der Final Fantasy-Serie legen großen Wert auf Story und Präsentation.
\\Leider fehlt uns sowohl Zeit, als auch Budget für derartige Projekte, weshalb wir uns eher an Indie-Projekten auf dem PC und "kleineren" Projekten auf den Konsolen orientieren. Das hoch gelobte the Banner Saga hat ein ausgereiftes Kampfsystem und wurde am Anfang von drei Leuten programmiert. Andere sogenannte Taktik-Rollenspiele wie Disgaea und Final Fantasy Tactics haben reduziertere Grafik und Story und konzentrieren sich eher auf vielschichtige Level-Systeme. Unser Ziel war es daher eher strategische Kämpfe und einen deutlich merkbaren Level-Fortschritt zu generieren, Dinge, die Pokémon Go fehlen.
\newpage
\subsubsection{5.2.1 The Banner Saga}
The Banner Saga ist ein von Stoic entwickeltes Taktik-Rollenspiel. Es wurde von drei ehemaligen BioWare-Mitarbeitern produziert, die 

\end{document}